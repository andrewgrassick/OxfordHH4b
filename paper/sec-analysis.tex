
\section{Analysis strategy}
\label{sec:analysis}

In this section we describe our analysis strategy, and in particular
the classification of each individual event into
different categories according to the event topology.

The starting point is clustering all final state particles in each
event,  after the parton shower (but before hadronization).
%
We adopt the following terminology for the different types of jets
that we will use in this work
\begin{itemize}
\item {\it Small-$R$ jets} are jets constructed with the
  anti-$k_T$ clustering algorithm with $R=0.4$ radius.
  %
  Only jets with transverse momentum $p_T > 40$~GeV within
  the central acceptance $|\eta|<2.5$ can be considered.

\item {\it Small-$R$ subjets} are anti-$k_T$ $R=0.3$ jets.
%
  They are required to have $p_T > 50$~GeV and $|\eta|<2.5$.

\item {\it Large-$R$ jets}  are jets constructed with the
  anti-$k_T$ clustering algorithm with $R=1.0$ radius.
  %
  We require that these large-$R$ jets satisfy  $p_T > 200$~GeV and
  $|\eta|<2.0$.
  %
  The smaller range in pseudo-rapidity, as compared to the small-$R$ jets,
  is motivated by mimicking the  experimental requirements
  related to the track-jet-based calibration).
  %
  On top of the basic $p_T$ and $\eta$ cuts, only large-$R$ jets
  that satisfy the BDRS mass-drop tagger condition will be kept, else
  the jet is discarded.
  %
  For the BDRS mass-drop tagger, we use the {\tt FastJet} default
  parameters of  $\mu = 0.67$ and $y_{\textrm{cut}}= 0.09$.
  \end{itemize}


Small-$R$ jets are $b$-tagged as follows: True $b$-jets, which have at least one $b$-quark among their constituents, have a 80\% $b$-tagging probability. For all other jets, a mistag rate of 0.01 is used.

Large-$R$ jets are $b$-tagged by ghost-associating anti-$k_T$ $R=0.3$ (selected as above) to the large-$R$ jets. A large-$R$ jets is considered $bb$-tagged if the leading and subleading $p_T$ associated subjets are $b$-tagged.



The resolved selection requires the presence of at least four $b$-tagged, selected anti-$k_T$ $R=0.4$ jets.

The di-Higgs system is reconstructed by considering all possibilities of forming two pairs of jets with invariant masses $m_{j1j2}$ and 
$m_{j3j4}$, respectively, and choosing the configuration that minimizes their difference $|m_{j1j2} - m_{j3j4}|$. Only the four leading-$p_T$ jets
are considered here.


The boosted selection requires the presence of at least two $bb$-tagged, selected anti-$k_T$ $R=1.0$ jets which are taken as the two Higgs boson candidates.


The intermediate topology requires the presence of at least one $bb$-tagged, selected anti-$k_T$ $R = 1.0$ jet. If there is more than one, the leading one is taken as the first Higgs boson candidate. In addition, there must be at least two anti-$k_T$ $R=0.4$ jets with a separation $\Delta R > 1.0$ with respect to the large-$R$ jet
that is selected as a Higgs boson candidate. The second Higgs boson candidate is reconstructed from the two small-$R$ jets that minimize the difference $|m_{fj} - m_{j1j2}|$.


All Higgs candidates are required to meet the following Higgs mass window cut: $|m_{H} - 125| < 40$~GeV

{\bf BELOW: outline of the section}\\


Describe the 3 different kinematic regimes: resolved, semi-boosted and boosted and selection cuts. Show table with numbers. How often do events end up in these categories.

Table 1: how many events we see from Signal and background in each category. Indicate also the overlap between the categories. This table should have also the S/B and S/sqrtB numbers. We need also the S/B and S/sqrtB numbers without the 2b2j and 4j backgrounds in order to compare with the UCL paper in this table.

Figure 1: Higgs pt, eta distributions

Figure 2: b-jet pt distributions

Figure 3: delta R distributions

Figure 4: mass distributions

Table 2: S/B and S/sqrtB table with
All these figures should have the backgrounds overlaid.

This section would show that the 2b2j and 4j backgrounds are important and that control of the fakes is necessary.

This section should also identify which of the different regimes is the most important for the measurement.

