
\section{Analysis strategy}
\label{sec:analysis}


In section we describe our analysis strategy, and in particular
the classification of each individual event into
different categories according to the event topology.

{\bf needs flow chart}

First of all we discuss the settings and nomenclature that
we will use for jet clustering in this work.
%
Then we move to discuss how $b$-tagging is simulated,
following closely the ATLAS techniques.
%
Then we discuss the categorisation of events between different
topologies, and how we can prioritise among them.


\subsection{Jet reconstruction}

In our analysis we will use different types of jet definitions,
depending of the type of analysis that needs to be performed.
%
Therefore, we being by enumerating the choices that we make.
%
The starting point after the parton shower is the clustering
of all final state particles.
%
In this work all the jet reconstruction algorithms
are obtained from the {\tt FastJet} program~\cite{Cacciari:2011ma},
v3.1.0.
%
We adopt the following terminology for the different types of jets
that we will use in this work:

\begin{itemize}
\item {\it Small-$R$ jets}.

  These are jets  reconstructed with the
  anti-$k_T$ clustering algorithm~\cite{Cacciari:2008gp} with $R=0.4$ radius.
  %
  When reconstructing small-$R$ jets, we will keep only
  those with transverse momentum $p_T \ge 40$~GeV
  and pseudo-rapidity $|\eta|<2.5$ within the central
  acceptance of ATLAS and CMS, jets that fail to satisfy these
  conditions are discarded

  They are required to have $p_T > 50$~GeV and $|\eta|<2.5$.

\item {\it Large-$R$ jets}.

  These jets are also constructed with the
  anti-$k_T$ clustering algorithm, now using a $R=1.0$ radius.
  %
  In order to be accepted, these large-$R$ jets should
  satisfy  $p_T \ge 200$~GeV and lie in a pseudo-rapidity range by
  $|\eta|<2.0$.
  %
  If these conditions are not satisfied, the jet is discarded.
  %
  The more restrictive range  in pseudo-rapidity
  as compared to the small-$R$ jets,
  is motivated by mimicking the  experimental requirements
  in ATLAS and CMS
  related to the track-jet-based calibration.

  In addition to the basic $p_T$ and $\eta$
  selection requirements, large-$R$ jets should also
  satisfy the  BDRS mass-drop tagger~\cite{Butterworth:2008iy}
  conditions, in order
  to enhance the discrimination of the signal over the background
  events.
  %
  For the BDRS mass-drop tagger, we use the {\tt FastJet} default
  parameters of  $\mu = 0.67$ and $y_{\textrm{cut}}= 0.09$.


  
\item {\it Small-$R$ subjets}.

  Once we have reconstructed a suitable large-$R$ jet, important
  information can be obtained by looking at the kinematics of
  the small-$R$ subjets.
  %
  These subjets will be obtained by reclustering the constituents
  of the large-$R$ jet, always with the  anti-$k_T$ algorithm,
  but this time with a smaller radius parameter $R=0.3$.
  %
  As we will discuss below, these small-$R$ subjets are the key of
  the $b$-tagging in the boosted category.
%
  \end{itemize}


\subsection{$b$-tagging}

Given the final state that needs to be reconstructed, the
optimisation of the $b$-tagging capabilities of the
LHC experiments is an essential requirement.
%
The $b$-tagging used in this feasibility study is inspired
in the current and projected ATLAS settings, and below
we also comment on the differences and similarities with the
corresponding CMS strategy.
%
For each type of jet defined above, a different
$b$-tagging strategy is used, as we discuss now.

\begin{itemize}

\item {\it Small-$R$ jets}.

  These are $b$-tagged as follows.
  %
  If a small-$R$ jet has at least one $b$-quark among their constituents,
  it will be tagged as a $b$-jet with probability $f_b$.
  %
  If no $b$-quarks are found among the constituents
  of this jet, it can be still be tagged as a $b$-jet with
  a mistag rate of $f_l$.
  %
  Note that the probability of tagging a jet is not increased
  if more than one $b$-quark is found among the jet constituents.

  \item {\it Large-$R$ jets}.

    Large-$R$ jets are $b$-tagged by ghost-associating anti-$k_T$ $R=0.3$
    subjets (defined as above) to the original large-$R$ jets.
    %
    In particular,
    a large-$R$ jet is considered  double-$b$-tagged if both
    the leading and subleading subjets (where the ordering
    is done in the subjet $p_T$) are both individually $b$-tagged.
    %
    Note that we only attempt to $b$-tag the two leading subjets:
    trying to $b$-tag all subjets above the $p_T$ cut degrades
    the signal significance due to the increase in combinatorics.

    Therefore, in this work we will exploit the capabilities of
    the LHC experiment of double-$b$-tagging inside large-$R$ jets.
    %
    Below we comment on the dependence of our results if single-$b$-tagging
    would be performed also for the large-$R$ jets, as done
    for the small-$R$ jets.

    Concerning the $b$-tagging probabilities and the
    light jet mistag probabilities, there are taken
    the same as for small-$R$ jets.
    %
    Therefore, a large-$R$ jet where the two leading
    subjets have at least one $b$-quark will be tagged
    with probability $f_b^2$, if only one of the two leading
    subjets has a $b$-quark then the tagging probability is
    $2f_bf_l$, and if none of the two have $b$-quarks
    as constituents, the mistag rate will be
    $f_l^2$.


\end{itemize}

Concerning the actual values of the $b$-tag probability $f_b$ and
the $b$-mistag probability of light jets $f_l$, in this work
we adopt as baseline the values $f_b=0.8$ and $f_l=0.01$.
%
In Sect.~\ref{sec:optimisation} we study the dependence of
our results against variations of $f_b$ and $f_l$ both
using more conservative and more aggressive values.
%
There we also study the dependence of the results
on other analysis settings related to the performance of
the experiments, for example by varying the momentum smearing.

\subsection{Event categorisation and cut flow}

In Table~\ref{sec:categorisation} we summarize the definitions of each
category


The di-Higgs system is reconstructed by considering all possibilities of forming two pairs of jets with invariant masses $m_{j1j2}$ and 
$m_{j3j4}$, respectively, and choosing the configuration that minimizes their difference $|m_{j1j2} - m_{j3j4}|$. Only the four leading-$p_T$ jets
are considered here.


The boosted selection requires the presence of at least two $bb$-tagged, selected anti-$k_T$ $R=1.0$ jets which are taken as the two Higgs boson candidates.


The intermediate topology requires the presence of at least one $bb$-tagged, selected anti-$k_T$ $R = 1.0$ jet. If there is more than one, the leading one is taken as the first Higgs boson candidate. In addition, there must be at least two anti-$k_T$ $R=0.4$ jets with a separation $\Delta R > 1.0$ with respect to the large-$R$ jet
that is selected as a Higgs boson candidate. The second Higgs boson candidate is reconstructed from the two small-$R$ jets that minimize the difference $|m_{fj} - m_{j1j2}|$.


The resolved selection requires the presence of at least four $b$-tagged, selected anti-$k_T$ $R=0.4$ jets.

All Higgs candidates are required to meet the following Higgs mass window cut: $|m_{H} - 125| < 40$~GeV


%%%%%%%%%%%%%%%%%%%%%%%%%%%%%%%%%%%
\begin{table}[h]
  \centering
  \small
  \begin{tabular}{c|c|c|c}
    \hline
    &  Boosted  &  Intermediate  &  Resolved  \\
    \hline
    \hline 
    Jets  &  $\ge 2$ large-$R$ jets  & 1 large-$R$ jet, $\ge 2$ small-$R$
    jets  &  $\ge$ 4 small-$R$ jets \\
    Cuts on jets  & $p_T \ge 200$ GeV   &  $p_T \ge 200$ GeV (large-$R$ jet)
    &  $p_T \ge 40$ GeV \\
    &  & $p_T \ge 40$ GeV (small-$R$ jets)   &    \\
    &   $|\eta|<2.0$  & $|\eta|<2.0$ (large-$R$ jet) & $|\eta|<2.5$ \\
    &        & $|\eta|<2.5$ (small-$R$ jet)  &  \\
    \hline
    \end{tabular}
  \caption{\small Definition of the three event categories used in this
    work, together with the corresponding basic kinematical cuts.
\label{sec:categorisation}
  }
\end{table}
%%%%%%%%%%%%%%%%%%%%%%%%%%%%%%%%%%%%%%

Describe the 3 different kinematic regimes: resolved, semi-boosted and boosted and selection cuts. Show table with numbers. How often do events end up in these categories.

Table 1: how many events we see from Signal and background in each category. Indicate also the overlap between the categories. This table should have also the S/B and S/sqrtB numbers. We need also the S/B and S/sqrtB numbers without the 2b2j and 4j backgrounds in order to compare with the UCL paper in this table.

Figure 1: Higgs pt, eta distributions

Figure 2: b-jet pt distributions

Figure 3: delta R distributions

Figure 4: mass distributions

Table 2: S/B and S/sqrtB table with
All these figures should have the backgrounds overlaid.

This section would show that the 2b2j and 4j backgrounds are important and that control of the fakes is necessary.

This section should also identify which of the different regimes is the most important for the measurement.

