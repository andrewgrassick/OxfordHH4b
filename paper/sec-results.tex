%%%%%%%%%%%%%%%%%%%%%%%%%%%%%%%%%%%%%%%%%%

\section{Cut-based analysis results}

\label{sec:results}

In this section we discuss the results of the traditional
cut-based analysis, using the strategy presented in the previous
section.
%
We compare our results with other recent related studies,
and in particular we study how the signal significance
is modified if only the $4b$ QCD background is taken into account,
but the $2b2j$ and $jjjj$ backgrounds are neglected, as done
in previous work.
%
Then in the next section we show how using multivariate techniques
one can substantially improve the signal significance as compared
to the cut-based analysis.



\subsection{Signal significance}

We now study the basic results of our cut based analysis, with
all backgrounds properly taken into account.

We begin by showing the cut-flow at various levels of the analysis.
%
Here we consider all backgrounds at the same time, and below
we comment on the role of the individual backgrounds.

We consider the following levels of the cut-flow
\begin{itemize}
\item {\bf C0}: these are the cross-sections at the generator
  level, before any analysis cuts
\item {\bf C1}: results after the basic jet selection
  and related cuts, but before $b$-tagging.
\item {\bf C2}: results after the $b$-tagging, but
  before applying the invariant mass window cut
\item {\bf C3}: final results of the cut-based
  analysis
  \end{itemize}
We consider the results for the individual categories, using the
exclusive priorisation discussed in the previous section.
%
In Appendix.~\ref{sec:overlap} we comment
interplay of the overlap between the various categories.
%
At each stage of the cut flow, we also provide the number
of events that would be expected at the HL-LHC with
an integrated luminosity of $\mathcal{L}=3000$ fb$^{-1}$,
the signal significance $S/\sqrt{B}$ and the signal
over background ratio $S/B$.
%
Results are shown both for the signal and for all backgrounds
combined.
%
The results for the three categories have been collected in
Table~\ref{table:cutflowboosted}.

We emphasize that it is important not only to achieve a good signal
significance, $S/sqrt{B}$, but also a good signal over background ratio $S/B$:
if the latter is too small, a measurement of the background with an unfeasibly small
systematic uncertainty would be required.

Let's begin with the discussion of the boosted category, that as we have explained
is the one that benefits from a higher signal significance.
%
We see that after all the cuts, including $b$-tagging,
we end up with almost 200 signal events at the HL-LHC, and
still a substantial background of around $40k$ events.
%
The signal significance is around one at the end of the
cut-based analysis.
%
Note that it is easy to compute the corresponding numbers
for the end of Run II at the LHC with
$\mathcal{L}_{\rm int}=300$ fb$^{-1}$: we have only
19 signal events, and the signal significance drops down to
$S/\sqrt{B}\sim 0.3$.
%
We will discuss in the next section what are the Run II prospects
once we include the effects of the MVA in the analysis.

%%%%%%%%%%%%%%%%%%%%%%%%%%%%%%%%%%%%%%%%%%%%%%%%%%%%%
\begin{table}[t]
  \centering
  \begin{tabular}{c|c|c||c|c||c|c}
    \hline
    \multicolumn{7}{c}{Boosted category}\\
    \hline
    \hline
    &    \multicolumn{2}{c||}{$\sigma$ (fb)}   &  \multicolumn{2}{c||}{$N_{\rm ev}$}
    &   $S/\sqrt{B}$  & $S/B$\\
      &    Signal & Back   &  Signal  & Back
    &   & \\
    \hline
        {\bf C0}  &  36.9  & $9.9\,10^{9}$ & $1.1\,10^5$ & $3.0\,10^{13}$  &  0.020 & $3.7\,10^{-9}$\\
        {\bf C1}  &  0.46    & $1.3\,10^6$    &  $1.4\,10^3$   & $3.9\,10^9$     & 0.022     &  $3.5\,10^{-7}$ \\
        {\bf C2}  &  0.064     &  12.6     &  193   &  $3.8\,10^4$    &  0.99    &  $5.1\,10^{-3}$ \\
        \hline
  \end{tabular}
  $\,$\\
  \vspace{0.4cm}
  \begin{tabular}{c|c|c|c|c|c|c}
    \hline
    \multicolumn{7}{c}{Intermediate category}\\
    \hline
    \hline
    &    \multicolumn{2}{c|}{$\sigma$ (fb)}   &  \multicolumn{2}{c|}{$N_{\rm ev}$}
    &   $S/\sqrt{B}$  & $S/B$\\
      &    Signal & Back   &  Signal  & Back
    &   & \\
    \hline
         {\bf C0}  &  36.9  & $9.9\,10^{9}$ & $1.1\,10^5$ & $3.0\,10^{13}$  &  0.020 & $3.7\,10^{-9}$\\
        {\bf C1}  &   1.44    &   $1.1\,10^{7}$  &  $4.3\,10^3$   &  $3.3\,10^{10}$    &  0.024    &  $1.3\,10^{-7}$ \\
        {\bf C2}  &   0.22    &  722   &  660   &   $2.2\,10^6$   &   0.45   &  $3.1\,10^{-4}$ \\
        \hline
  \end{tabular}
  $\,$\\
  \vspace{0.4cm}
  \noindent
  \begin{tabular}{c|c|c|c|c|c|c}
    \hline
    \multicolumn{7}{c}{Resolved category}\\
    \hline
    \hline
    &    \multicolumn{2}{c|}{$\sigma$ (fb)}   &  \multicolumn{2}{c|}{$N_{\rm ev}$}
    &   $S/\sqrt{B}$  & $S/B$\\
      &    Signal & Back   &  Signal  & Back
    &   & \\
    \hline
       {\bf C0}  &  36.9  & $9.9\,10^{9}$ & $1.1\,10^5$ & $3.0\,10^{13}$  &  0.020 & $3.7\,10^{-9}$\\
        {\bf C1}  &   3.34    & $7.5\,10^{7}$    & $1.1\,10^4$    & $3.3\,10^{10}$     & 0.021     & $4.5\,10^{-8}$  \\
        {\bf C2}  &   0.54    &  $3.5\,10^{3}$   &  $1.6\,10^{3}$   &   $1.1\,10^{7}$   & 0.50     &  $1.5\,10^{-4}$ \\
        \hline
  \end{tabular}
  \caption{\small Cut-flow for the analysis of the boosted (top),
    intermediate (middle) and resolved (bottom)
    categories.
    %
    At each level of the cut-flow, we indicate the cross-sections and the number of
    expected events at the HL-LHC for $\mathcal{L}_{\rm int}=3$ ab$^{-1}$, both for
    signal events and for all the backgrounds combined.
    %
    We also provide in each case the
    signal significance $S/\sqrt{B}$ and the signal
    over background ratio $S/B$.
    %
    The first row {\bf C0} is the generator-level result and thus is common
    to all three categories.
    %
    See text for more details.
    \label{table:cutflowboosted}
  }
\end{table}
%%%%%%%%%%%%%%%%%%%%%%%%%%%%%%%%%%%%%%%%%%%%%%%%%%%%%

Both the intermediate and resolved categories benefit from higher signal yields,
specially in the resolved category, but this enhancement is cancellend by the stronger
increase in the QCD multijet background.
%
We see that in both cases the signal significance is around half of that in the boosted category,
with in addition $S/B$ being an order of magnitude smaller.
%
Therefore, the boosted category is clearly determined to be the most useful category,
specially due to the significant suppression of the QCD multijet background.
%
And we still have not exploited all the rich information contained on the jet
substructure, as we will do in the next section.

From the results of Table~\label{table:cutflowboosted}
we see that after the cut-based analysis, the significance of the Higgs pair production
observation in the $4b$ channel using the boosted topology is ... and that the combination
of the boosted, intermediate and resolved categories gives ....
%
We discuss in the next section of this significance can be enhanced by means
of multivariate analysis.

\subsection{Comparison with previous work}

We now compare with two recent studies that have also studied the
feasibility of SM Higgs pair production in the $4b$ final state,
those of the UCL group~\cite{Wardrope:2014kya} and of the
Durham group~\cite{deLima:2014dta}.
%
In order to perform a meaningful comparison, we will only consider
here the $4b$ QCD and $t\bar{t}$ backgrounds, as was done
in these two studies.

{\bf paragraph about UCL strategy}


{\bf paragraph about Durham strategy}


In order to compare with previous results, it is interesting to
study how our results change if we consider only the QCD $4b$
multi-jet background, but ignore the $2b2j$ and $4j$ backgrounds,
that is, we ignore the contribution from the fakes.
%
For the boosted category, the results are shown in
Table~\ref{table:cutflowboosted4B}, which is the analog of
Table~\ref{table:cutflowboosted} but with only the QCD
$4b$ background taken into account.
%
As we can see, these results indicate that given the similarity of the final states
in signal and background events in these cases, the signal significance is
relatively stable at various stages of the cut-flow.
%
As compared to the case in which all backgrounds are taken into account, we
now get a factor two smaller backgrounds: we conclude that in the boosted category
the effect of the fakes is non-negligible, with their contribution being
comparable to that of the irreducible $4b$ QCD multi-jet background.


%%%%%%%%%%%%%%%%%%%%%%%%%%%%%%%%%%%%%%%%%%%%%%%%%%%%%
\begin{table}[t]
  \centering
  \begin{tabular}{c|c|c||c|c||c|c}
    \hline
    \multicolumn{7}{c}{Boosted category}\\
    \hline
    \hline
    &    \multicolumn{2}{c||}{$\sigma$ (fb)}   &  \multicolumn{2}{c||}{$N_{\rm ev}$}
    &   $S/\sqrt{B}$  & $S/B$\\
      &    Signal & Back   &  Signal  & Back
    &   & \\
    \hline
        {\bf C0}  &  36.9  & $1.1\,10^{6}$ & $1.1\,10^5$ & $3.3\,10^{9}$  &  1.90 & $3.3\,10^{-5}$\\
        {\bf C1}  &  0.46    & $1.0\,10^2$    &  $1.4\,10^3$   & $3.1\,10^5$     & 2.48     &  $4.4\,10^{-3}$ \\
        {\bf C2}  &  0.064     &  7.1     &  193   &  $2.1\,10^4$    &  1.30    &  $9.1\,10^{-3}$ \\
        \hline
  \end{tabular}
  \caption{\small Same as Table~\ref{table:cutflowboosted}, but now
    only with the QCD $4b$ multi-jet background taken into account.
    %
    \label{table:cutflowboosted4B}
  }
\end{table}
%%%%%%%%%%%%%%%%%%%%%%%%%%%%%%%%%%%%%%%%%%%%%%%%%%%%%
