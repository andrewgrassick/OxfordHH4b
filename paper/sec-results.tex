%%%%%%%%%%%%%%%%%%%%%%%%%%%%%%%%%%%%%%%%%%

\section{Cut-based analysis results}

\label{sec:results}

In this section we discuss the results of the traditional
cut-based analysis, using the strategy presented in the previous
section.
%
We compare our results with other recent related studies,
and in particular we study how the signal significance
is modified if only the $4b$ QCD background is taken into account,
but the $2b2j$ and $jjjj$ backgrounds are neglected, as done
in previous work.
%
Then in the next section we show how using multivariate techniques
one can substantially improve the signal significance as compared
to the cut-based analysis.



\subsection{Signal significance}

We now study the basic results of our cut based analysis, with
all backgrounds properly taken into account.

We begin by showing the cut-flow at various levels of the analysis.
%
Here we consider all backgrounds at the same time, and below
we comment on the role of the individual backgrounds.

We consider the following levels of the cut-flow
\begin{itemize}
\item {\bf C0}: these are the cross-sections at the generator
  level, before any analysis cuts
\item {\bf C1}: results after the basic jet selection
  and related cuts, but before $b$-tagging.
\item {\bf C2}: results after the $b$-tagging, but
  before applying the invariant mass window cut
\item {\bf C3}: final results of the cut-based
  analysis
  \end{itemize}
We consider the results for the individual categories, using the
exclusive priorisation discussed in the previous section.
%
In Appendix.~\ref{sec:overlap} we comment
interplay of the overlap between the various categories.
%
At each stage of the cut flow, we also provide the number
of events that would be expected at the HL-LHC with
an integrated luminosity of $\mathcal{L}=3000$ fb$^{-1}$,
the signal significance $S/\sqrt{B}$ and the signal
over background ratio $S/B$.
%
Results are shown both for the signal and for all backgrounds
combined.

Let's begin with the discussion of the boosted category,
Table~\ref{table:cutflowboosted}, that as we have explained
is the one that benefits from a higher signal significance.

%%%%%%%%%%%%%%%%%%%%%%%%%%%%%%%%%%%%%%%%%%%%%%%%%%%%%
\begin{table}[t]
  \centering
  \begin{tabular}{c|c|c|c|c|c|c}
    \hline
    \multicolumn{7}{c}{Boosted category}\\
    \hline
    \hline
    &    \multicolumn{2}{c|}{$\sigma$ (fb)}   &  \multicolumn{2}{c|}{$N_{\rm ev}$}
    &   $S/\sqrt{B}$  & $S/B$\\
      &    Signal & Back   &  Signal  & Back
    &   & \\
    \hline
        {\bf C0}  &  AAA  & AAA & AAA & AAA  &  AAA & AAA\\
        {\bf C1}  &       &     &     &      &      &   \\
        {\bf C2}  &       &     &     &      &      &   \\
        {\bf C3}  &       &     &     &      &      &   \\
        \hline
  \end{tabular}
  \caption{\small Cut-flow for the analysis of the boosted
    category.
    %
    At each level we indicate the cross-sections and the number of
    expected events at the HL-LHC both for
    signal and for all the backgrounds combined,
    as well as the signal significance $S/\sqrt{B}$ and the signal
    over background ratio $S/B$.
    \label{table:cutflowboosted}
  }
\end{table}
%%%%%%%%%%%%%%%%%%%%%%%%%%%%%%%%%%%%%%%%%%%%%%%%%%%%%

%%%%%%%%%%%%%%%%%%%%%%%%%%%%%%%%%%%%%%%%%%%%%%%%%%%%%
\begin{table}[t]
  \centering
  \begin{tabular}{c|c|c|c|c|c|c}
    \hline
    \multicolumn{7}{c}{Intermediate category}\\
    \hline
    \hline
    &    \multicolumn{2}{c|}{$\sigma$ (fb)}   &  \multicolumn{2}{c|}{$N_{\rm ev}$}
    &   $S/\sqrt{B}$  & $S/B$\\
      &    Signal & Back   &  Signal  & Back
    &   & \\
    \hline
        {\bf C0}  &  AAA  & AAA & AAA & AAA  &  AAA & AAA\\
        {\bf C1}  &       &     &     &      &      &   \\
        {\bf C2}  &       &     &     &      &      &   \\
        {\bf C3}  &       &     &     &      &      &   \\
        \hline
  \end{tabular}
  \caption{\small Same as Table~\ref{table:cutflowboosted} now for
    the intermediate category.
    \label{table:cutflowintermediate}
  }
\end{table}
%%%%%%%%%%%%%%%%%%%%%%%%%%%%%%%%%%%%%%%%%%%%%%%%%%%%%


%%%%%%%%%%%%%%%%%%%%%%%%%%%%%%%%%%%%%%%%%%%%%%%%%%%%%
\begin{table}[t]
  \centering
  \begin{tabular}{c|c|c|c|c|c|c}
    \hline
    \multicolumn{7}{c}{Intermediate category}\\
    \hline
    \hline
    &    \multicolumn{2}{c|}{$\sigma$ (fb)}   &  \multicolumn{2}{c|}{$N_{\rm ev}$}
    &   $S/\sqrt{B}$  & $S/B$\\
      &    Signal & Back   &  Signal  & Back
    &   & \\
    \hline
        {\bf C0}  &  AAA  & AAA & AAA & AAA  &  AAA & AAA\\
        {\bf C1}  &       &     &     &      &      &   \\
        {\bf C2}  &       &     &     &      &      &   \\
        {\bf C3}  &       &     &     &      &      &   \\
        \hline
  \end{tabular}
  \caption{\small Same as Table~\ref{table:cutflowboosted} now for
    the boosted category.
    \label{table:cutflowresolved}
  }
\end{table}
%%%%%%%%%%%%%%%%%%%%%%%%%%%%%%%%%%%%%%%%%%%%%%%%%%%%%

From the results of Tables~\label{table:cutflowboosted}--\ref{table:cutflowresolved}
we see that after the cut-based analysis, the significance of the Higgs pair production
observation in the $4b$ channel using the boosted topology is ... and that the combination
of the boosted, intermediate and resolved categories gives ....
%
We discuss in the next section of this significance can be enhanced by means
of multivariate analysis.

\subsection{Comparison with previous work}

We now compare with two recent studies that have also studied the
feasibility of SM Higgs pair production in the $4b$ final state,
those of the UCL group~\cite{Wardrope:2014kya} and of the
Durham group~\cite{deLima:2014dta}.
%
In order to perform a meaningful comparison, we will only consider
here the $4b$ QCD and $t\bar{t}$ backgrounds, as was done
in these two studies.

{\bf paragraph about UCL strategy}


{\bf paragraph about Durham strategy}
