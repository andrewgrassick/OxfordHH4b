\section{Conclusions and outlook}
\label{sec:conclusions}

In this work we have presented a feasibility study for
 the measurement of Higgs pair production in the $b\bar{b}b\bar{b}$
final state at the High-Luminosity LHC with $\mathcal{L}=3$ ab$^{-1}$.
%
Our strategy is based on the combination of traditional
cut-based analysis with state-of-the-art multivariate techniques.
%
We take into account 
all relevant backgrounds, in particular
the irreducible $4b$
and the reducible 
$2b2j$ and $4j$ QCD multijets.
%
We have illustrated how the $2b2j$ component leads to
a contribution comparable to that of the $4b$ process,
due to a combination of  parton shower effects, $b$-quark 
pair radiation, and selection requirements.
%
We have also demonstrated the robustness of our analysis strategy
under the addition of significant pileup.

Combining the contributions from the resolved,
intermediate and boosted categories, we find that, for
$\mathcal{L}=3$ ab$^{-1}$, the
signal significance for
the production of Higgs pairs can be as large as $S/\sqrt{B}\simeq 4.0$.
%
This indicates that, already from the $b\bar{b}b\bar{b}$
final state alone,
it should be possible to claim evidence for Higgs pair production at
the HL-LHC.
%
Our study also suggests possible avenues that the LHC experiments
could explore to further improve this signal significance.
%
One handle would be to reduce the contribution from light and charm
jet mis-identification, ensuring that the irreducible $4b$ background 
 dominates over the $2b2j$ component.
%
Another possibility would be to improve the mass resolution of the Higgs
reconstruction
in high-PU environments, and, more in general,
to optimize the PU subtraction
strategy in order
to reduce the impact of PU in the modelling
of kinematical variables and the associated
degradation in the MVA discrimination.
%
Such experimental improvements might
even make
an observation of Higgs pair production possible at the
end of Run II with
$\mathcal{L}=300$ fb$^{-1}$.

One important implication of this work is that it should
be possible to improve  the accuracy on the extraction of
the Higgs trilinear coupling $\lambda$ from
a measurement of the
$\sigma\lp hh\to b\bar{b}b\bar{b}\rp$ cross-section, as compared
to existing estimates.
%
A determination of $\lambda$ in our approach is however
rather
non-trivial, involving
 not only regenerating signal samples
 for a wide range of values of  $\lambda$, but also
 repeating the analysis
optimisation, including the MVA training, for each
of these values.
%
This study is left to a future
publication, where we will also
compare with the corresponding  precision 
that has been reported from other final states such as
 $b\bar{b}\gamma\gamma$
and $b\bar{b}\tau\tau$.
%
It will also be  interesting to perform
this exercise for a 100 TeV hadron collider~\cite{Barr:2014sga,
  Azatov:2015oxa,Papaefstathiou:2015iba,
  Arkani-Hamed:2015vfh}.
%
While signal yields will be certainly increased, also the (gluon-driven) QCD
multijet background will grow strongly.
%
Revisiting
the present analysis, including the MVA optimization,
at 100 TeV will allow us
to assess the accuracy of an extraction of the trilinear
coupling $\lambda$ from the $b\bar{b}b\bar{b}$ final state
at 100 TeV.


Our strategy relies on the modeling of the kinematical
distributions of signal and background events, since these provide
the inputs to the MVA discriminant.
%
In this respect, it would be important, having established the key
relevance of the $b\bar{b}b\bar{b}$ channel for the study of
Higgs pair production, to revisit and improve the
theoretical modeling of our signal and background simulation,
in particular using NLO calculations matched to
parton showers both for signal~\cite{Frederix:2014hta,Maierhofer:2013sha}
and for backgrounds~\cite{Alwall:2014hca,Gleisberg:2008ta}.
%

In this work we have considered only the SM production mechanism,
but many BSM scenarios predict deviations
in Higgs pair production, both at the level of total rates
and of
differential distributions.
%
In the absence of new explicit degrees of freedom,
deviations from the SM can be parametrized in
the EFT framework using higher-order
operators~\cite{Azatov:2015oxa,Goertz:2014qta}.
%
Therefore, we plan to study the constraints
on the coefficients of these effective
operators that can be obtained from measurements
of various kinematical distributions
in the $hh\to b\bar{b}b\bar{b}$ process.
%
Note that the higher rates of the $b\bar{b}b\bar{b}$ final state as compared to
other final states, such as
$b\bar{b}\gamma\gamma$, allow for a better constraint upon operators
that modify the high-energy behavior
of the theory, for instance,
it would become possible
to access the tail of the $m_{hh}$ distribution.


As in the case of the extraction of the Higgs
trilinear coupling $\lambda$, such a study
would be a computationally intensive task, since
BSM dynamics will modify the shapes of the kinematical
distributions and thus in principle each point in the EFT parameter
space would require a re-optimization with a newly trained
MVA.
%
In order to explore efficiently the BSM parameters
without having to repeat the full analysis
for each point, modern statistical techniques
such as the Cluster Analysis method proposed
in Ref.~\cite{Dall'Osso:2015aia} might be helpful.



\bigskip
\bigskip
\begin{center}
\rule{5cm}{.1pt}
\end{center}
\bigskip
\bigskip

{\bf\noindent  Acknowledgments \\}
We thank F.~Bishara, R.~Contino, A.~Papaefstathiou and
G.~Salam for useful discussions on the topic
of Higgs pair production.
%
We thank E.~Vryonidou and M.~Zaro for
assistance with di-Higgs production
  in {\tt MadGraph5\_aMC@NLO}.
%
  J.~R. is supported by an STFC Rutherford Fellowship and
  Grant ST/K005227/1 and ST/M003787/1.
%
J.~R. and N.~H. are
supported by an European Research Council Starting Grant ``PDF4BSM".
%
K.~B. is supported by a Rhodes Scholarship.
%
D.~B., J.~F. and C.~I. are supported by the STFC.
