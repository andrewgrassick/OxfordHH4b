\section{Conclusions and outlook}
\label{sec:conclusions}

In this work, we have presented a detailed study of the feasibility
of the measurement of Higgs pair production in the $b\bar{b}b\bar{b}$
final state at the High-Luminosity LHC.
%
Our strategy is based on the combination of a traditional
cut-based analysis with modern multivariate analyses,
which allows improving the signal significance of this channel
as compared to previous studies.
%
Following a careful optimisation of the
signal reconstruction strategy for different event topologies,
we find that the highest significance is obtained
in the 
boosted category, while complementary information is also provided
by the intermediate and resolved categories.
%

Our analysis takes into account 
all relevant backgrounds, in particular
the dominant QCD multijets.
%
We include not only the irreducible $4b$ component, but
also the $2b2j$ and $4j$ backgrounds, which contribute
to the yield of reconstructed Higgs candidates
when light partons are mistagged as $b$-quarks.
%
We show that neglecting the contribution
from  fakes is a poor approximation, specially for the resolved and intermediate
categories.
%
The reason is that
the impact of fakes
can be substantially enhanced, when compared to a naive parton
level estimate, due to combinatorics and $b\bar{b}$
pair radiation during the parton shower.

A major difficulty in measurements involving relatively low $p_T$
jets at the HL-LHC is the large
amount of PU expected.
%
To account for this, we have simulated realistic PU conditions
and used the recently developed {\tt SoftKiller} method to reduce
the PU contamination.
%
We find that, especially in the boosted category, the impact of
PU after its subtraction is rather mild, and in particular the jet
substructure variables, which carry substantial
discrimination information, are reasonably robust against PU effects.


For a PU of $\la n_{\rm PU}\ra=150$, along what is expected
at the HL-LHC, our main result is that the combined
signal significance for the three categories is $S/\sqrt{B}\simeq 4.0$,
substantially improving previous studies.
%
Therefore, we conclude that, already from the $b\bar{b}b\bar{b}$
final state alone,
it should be possible to claim evidence for Higgs pair production at
the HL-LHC.
%
This  estimate for $S/\sqrt{B}$  could
be further enhanced,
in particular following an optimization of the PU subtraction
strategy.


One important implication of our study is that it should
be possible to improve  the accuracy on the extraction of
the Higgs trilinear coupling $\lambda$ from
a measurement of the
$\sigma\lp hh\to b\bar{b}b\bar{b}\rp$ cross-section, as compared
to existing estimates.
%
A determination of $\lambda$ in our approach would require
 however a substantial
 amount of work, involving
 not only regenerating signal samples
 for a wide range of values of  $\lambda$, but also
 repeating the analysis
optimisation, including the MVA training, for each
of these values.
%
This study is left to a future
publication, where we will also
compare with the corresponding  precision 
that has been obtained in other final states.
%
It will be in particular interesting to contrast
with the extraction of $\lambda$ from the
 $b\bar{b}\gamma\gamma$
and $b\bar{b}\tau\tau$ final states, that, while having
smaller rates as compared to $b\bar{b}b\bar{b}$,
benefit from reduced backgrounds.
%


Our analysis relies on the careful modeling of kinematical
distributions of signal and background events, since these provide
the inputs to the MVA discriminant.
%
In this respect, it would be important, having established the key
relevance of the $b\bar{b}b\bar{b}$ channel for the study of
Higgs pair production, to revisit and improve the
theoretical modeling of our signal and background simulation,
in particular using NLO calculations matched to
parton showers both for signal~\cite{Frederix:2014hta,Maierhofer:2013sha}
and for backgrounds~\cite{Alwall:2014hca,Gleisberg:2008ta}.
%

In this work we have considered only the SM production mechanism,
but many BSM scenarios predict deviations
in Higgs pair production, both at the level of total rates
and of
differential distributions.
%
In the absence of new explicit degrees of freedom,
deviations from the SM can be parametrized in
the EFT framework using higher-order
operators~\cite{Azatov:2015oxa,Goertz:2014qta}.
%
We thus plan to study which are the constraints
on the coefficients of these effective
operators that can be obtained from measurements
of various $hh\to b\bar{b}b\bar{b}$ distributions.
%
Note that the higher rates of the $b\bar{b}b\bar{b}$ final state as compared to
other final states, such as
$b\bar{b}\gamma\gamma$, allow to better constrain operators
that modify the high-energy behavior
of the theory, in particular
since it becomes possible
to access the tail of the $m_{hh}$ distribution.


As in the case of the extraction of the Higgs
trilinear coupling $\lambda$, such study
would be a computationally intensive task, since
BSM dynamics will modify the shapes of the kinematical
distributions, and thus in principle each point in the EFT parameter
space would require a re-optimization with a new training
of the MVA.
%
In order to explore efficiency the BSM parameters
without having to repeat the full analysis
for each point, modern techniques
such as the Cluster Analysis method proposed
in Ref.~\cite{Dall'Osso:2015aia} could be useful.

Finally, it would be interesting to estimate the prospects for
Higgs pair production in the $b\bar{b}b\bar{b}$  final
state at a Future Circular
Collider (FCC) with a center of mass energy of
100 TeV~\cite{Barr:2014sga,Azatov:2015oxa,Papaefstathiou:2015iba}.
%
While signal yields would be certainly increased, also the QCD
multijet background would grow, so it would necessary to repeat
our analysis, including full background simulation, at 100 TeV,
and assess if the accuracy in the extraction of the trilinear $\lambda$
is improved or degraded as compared to the 14 TeV case.
%



\bigskip
\bigskip
\begin{center}
\rule{5cm}{.1pt}
\end{center}
\bigskip
\bigskip

{\bf\noindent  Acknowledgments \\}
We thank R.~Contino, A.~Papaefstathiou and
G.~Salam for useful discussions on the topic
of Higgs pair production.
%
We thank E.~Vrodynou and M.~Zaro for assistance with di-Higgs production
  in {\tt MadGraph5\_aMC@NLO}.
%
  J.~R. is supported by an STFC Rutherford Fellowship and
  Grant ST/K005227/1 and ST/M003787/1.
%
J.~R. and N.~H. are
supported by an European Research Council Starting Grant ``PDF4BSM".
%
N.~B.~N. is supported by a PhD scholarship from the Malaysia
government.
%
K.~B. is supported by a studentships from the Rhodes Foundation.
%
J.~F. and C.~I. are supported by the STFC.
