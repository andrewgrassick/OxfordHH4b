\section{Conclusions and outlook}
\label{sec:conclusions}

In this work we have presented a detailed study of the feasibility
of the measurement of Higgs pair production in the $bb\bar{b}\bar{b}$
final state at the High-Luminosity LHC.
%
As opposed to previous works, we have combined a traditional
cut-based analysis with a modern multivariate analysis,
which allows one to significantly improve the signal significance.
%
A careful analysis of the different possible methods to reconstruct
Higgs pairs finds that the boosted category is the most efficient
one, but that additional complementary information can also be provided
by the intermediate and resolved categories.
%
Our analysis includes all relevant backgrounds, including the
effects from fake $b$-tags, neglected in previous works,
which turn out to be quite important specially
in the boosted and in the resolved categories.

Thanks for the MVA improvements, and combining the information
from the three exclusive categories, we find a signal significance
of 3.8.
%
Therefore, we can conclude that, already on the $4b$ channel along,
it should be possible to observe evidence for this process at
the HL-LHC.
%
A combination with other decay channels, like $b\bar{b}\gamma\gamma$
and $b\bar{b}\tau\tau$ would still be required in order
to claim discovery of Higgs pair production.


One important implication of our results is which is the accuracy
that can be achieved for a measurement of the
Higgs trilinear coupling $\lambda$ from a measurement
of the $4b$ final state.
%
We plan to quantify this in future work.
%
In addition,
in this work we have considered only SM production mechanism,
but it is well known that many BSM scenarios predict deviations
in Higgs pair production, both at the level of rates,
for instance when the trilinear coupling $\lambda$, and of
shapes of the distributions, when higher-order effective
operators are included in the theory~\cite{Azatov:2015oxa}.
%
In this respect, we plan to study which are the constraints
on BSM physics that can be obtained from measurements
of Higgs pair production in the $4b$ channel.
%
This however will be a computationally intensive task since in
general BSM physics modify the shapes of the Higgs kinematical
distributions, and thus each different point in the parameter
space would require a re-optimization via a new training
of the MVA.

Finally, it would be important to estimate the prospects for
Higgs pair production in the $4b$ at a Future Circular
Collider with a center of mass energy of 100 TeV, along the lines
of similar studies~\cite{Barr:2014sga}.
%
While signal yields would be certainly increased, also the QCD
multijet background would grow, so it is necessary to repeat
our analysis, including full background simulation, for 100 TeV,
in order to be able to draw any definite conclusion here.



\bigskip
\bigskip
\begin{center}
\rule{5cm}{.1pt}
\end{center}
\bigskip
\bigskip

{\bf\noindent  Acknowledgments \\}
We thank R.~Contino, A.~Papaefstathiou and
G.~Salam for useful discussions on the topic
of Higgs pair production.
%
We thank E.~Vrodynou for assistance with di-Higgs production
  in {\tt MadGraph5\_aMC@NLO}.
%
J.~R. is supported by an STFC Rutherford Fellowship ST/K005227/1.
%
J.~R. and N.~H. are
supported by an European Research Council Starting Grant "PDF4BSM".
%
N.~B.~N. is supported by a PhD scholarship from the Malaysia
government.
K.~B. is supported by a studentships from the Rhodes Foundation.
%
J.~F. and C.~I. are supported by the STFC.
