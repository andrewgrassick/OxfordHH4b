\section{Summary and outlook}
\label{sec:conclusions}

In this work we have presented a detailed study of the feasibility
of a measurement of Higgs pair production in the $b\bar{b}b\bar{b}$
final state at the High-Luminosity LHC.
%
Our strategy is based on the combination of a traditional
cut-based analysis with modern multivariate techniques,
which allows us to improve the signal significance of this channel
as compared to previous studies.
%
Following a careful optimisation of the
signal reconstruction strategy for different event topologies,
we find that the highest significance is obtained
in the 
boosted category, and that complementary information is also provided
by the intermediate and resolved categories.
%


Our analysis is based on a complete simulation of
all relevant backgrounds, in particular
of QCD multijets.
%
We include not only the irreducible $4b$ component, but
also the $2b2j$ and $4j$ backgrounds, which contribute
to the yield of reconstructed Higgs candidates
when light partons are mistagged as $b$-quarks.
%
We show that neglecting the contribution
from these fakes is a poor approximation
in general, specially for the resolved and intermediate
categories, and that the origin of this is that
the impact of fakes
is substantially enhanced when compared to a naive parton
level estimate due to combinatorics and $b\bar{b}$
pair radiation during the parton shower.

The combination of the three categories leads to a total
signal significance of $S/\sqrt{B}\simeq 3.2$.
%
Therefore, we can conclude that, already from the $4b$ channel alone,
it should be possible to claim evidence for Higgs pair production at
the HL-LHC.
%
We have estimated the robustness of these estimates by varying some
of the analysis settings, such as the $b$-quark tagging efficiency
or the momentum resolution, finding that one could achieve
a signal significance of up to $S/\sqrt{B}\simeq 4.2$ in
the most aggressive scenarios.
%
However, it is also clear that in order
to claim discovery of Higgs pair production
(with the traditional 5-$\sigma$ criterion)
a combination with other decay channels
would still be required.
%
In this respect, the 
 $b\bar{b}\gamma\gamma$
and $b\bar{b}\tau\tau$ final states are the most
promising, since while having
smaller rates as compared to $4b$
they
benefit from reduced backgrounds.
%
We also find that an observation of Higgs pair production at the
LHC Run II is very difficult unless the rates are enhanced
by BSM dynamics.

The main limitation of this work is the lack a full
detector simulation, and the assumption that the large PU
expected at the HL-LHC can be removed using
available background
suppression methods~\cite{Cacciari:2009dp,TheATLAScollaboration:2013pia,Butterworth:2008iy,Cacciari:2007fd,Krohn:2009th,Krohn:2013lba,Ellis:2009me,Bertolini:2014bba,Cacciari:2014gra,Cacciari:2014jta,Berta:2014eza,Larkoski:2014wba} without modifying significantly the
kinematic variables that we input to the MVA.
%
While these are reasonable approximations (and we mimic to some
extent the detector resolution with the smearing), they need
to be validated in a complete detector simulation with realistic
PU effects.
%
In addition, it would be desirable, having established the
value of the $4b$ channel for the study of
Higgs pair production, to revisit and improve the
theoretical modeling of signal and background simulation,
in particular using NLO calculations matched to
parton showers both for signal~\cite{Frederix:2014hta,Maierhofer:2013sha}
and for backgrounds~\cite{Alwall:2014hca,Gleisberg:2008ta}.
%
We plan to address these two issues in a follow-up
publication.

One important application of our results
would be to quantify the accuracy on the extraction of
the Higgs trilinear coupling $\lambda$
that could be achieved from
a measurement of the
$\sigma\lp hh\to 4b\rp$ cross-section.
%
This is however not straightforward since
the determination of the trilinear coupling
requires not only to regenerate signal samples
for a wide range values of  $\lambda$, but also to
redo the analysis
optimisation (including the MVA training) for each
of these values.
%
We plan to explore this possibility in the future, and to
compare with the precision in the determination of
$\lambda$ that has been claimed in other final states.

In addition,
in this work we have considered only the SM production mechanism,
but many BSM scenarios predict deviations
in Higgs pair production, both at the level of total rates
and of
differential distributions.
%
In the absence of new explicit degrees of freedom,
deviations from the SM can be parametrized in
the EFT framework using higher-order
operators~\cite{Azatov:2015oxa,Goertz:2014qta}.
%
We thus plan to study which are the constraints
on the coefficients of these effective
operators that can be obtained from measurements
of various $hh\to 4b$ distributions.
%
Note that the higher rates of the $4b$ final state as compared to
say $b\bar{b}\gamma\gamma$ allow to better constrain operators
that modify the high-energy behavior
of the theory, in particular
since it becomes possible
to access the tail of the $m_{hh}$ distribution.


As in the case of the extraction of $\lambda$, such study
would be a computationally intensive task, since
BSM dynamics will modify the shapes of the kinematical
distributions, and thus in principle each point in the EFT parameter
space would require a re-optimization with a new training
of the MVA.
%
In order to explore efficiency the BSM parameters
without having to redo a full analysis
for each point, modern techniques
such as the cluster analysis proposed
in Ref.~\cite{Dall'Osso:2015aia} should be very useful.

Finally, it would be important to estimate the prospects for
Higgs pair production in the $4b$  final
state at a Future Circular
Collider (FCC) with a center of mass energy of
100 TeV~\cite{Barr:2014sga,Azatov:2015oxa,Papaefstathiou:2015iba}.
%
While signal yields would be certainly increased, also the QCD
multijet background would grow, so it is necessary to repeat
our analysis, including full background simulation, at 100 TeV,
and assess if the accuracy in the extraction of the trilinear $\lambda$
is improved or degraded as compared to the 14 TeV case.
%
This study should address both SM and BSM
production mechanisms.


\bigskip
\bigskip
\begin{center}
\rule{5cm}{.1pt}
\end{center}
\bigskip
\bigskip

{\bf\noindent  Acknowledgments \\}
We thank R.~Contino, A.~Papaefstathiou and
G.~Salam for useful discussions on the topic
of Higgs pair production.
%
We thank E.~Vrodynou for assistance with di-Higgs production
  in {\tt MadGraph5\_aMC@NLO}.
%
J.~R. is supported by an STFC Rutherford Fellowship ST/K005227/1.
%
J.~R. and N.~H. are
supported by an European Research Council Starting Grant "PDF4BSM".
%
N.~B.~N. is supported by a PhD scholarship from the Malaysia
government.
K.~B. is supported by a studentships from the Rhodes Foundation.
%
J.~F. and C.~I. are supported by the STFC.
