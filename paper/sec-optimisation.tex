
\section{Optimisation}
\label{sec:optimisation}

One important motivation of this work was to quantify which aspects
of the performance of the LHC detectors needs to be improved
the most in order to maximise the significance of the observation
of double Higgs production in the $4b$ final state.
%
In this section we explore how the results with the baseline settings,
summarized in the previous section, are modified when some of
these settings are changed.
%
In particular we will explore the dependence of the results on
the $b$-tagging probability $f_b$, the light jet mistag rate
$f_l$, as well on the momentum smearing fraction.
%
The variations of the analysis
settings that we will explore are collected in
Table~\ref{sec:variations}.

%%%%%%%%%%%%%%%%%%%
\begin{table}[h]
  \centering
  \begin{tabular}{|c|c|c|c|}
\hline
    Scenario  &  $f_b$  &  $f_l$  &  $\Delta p_T$ \\
    \hline
    \hline
    Baseline  &  0.8   &   0.01  &  5\% \\
    \hline
    A        &  0.9   &   0.01  &  5\% \\
    B        &  0.7   &   0.01  &  5\% \\
    C        &  0.8   &   0.005  &  5\% \\
    D        &  0.8   &   0.02  &  5\% \\
    E        &  0.9   &   0.005  &  2\% \\   
    \hline
  \end{tabular}
  \caption{\small The baseline settings for the $b$-jet
    tagging probability, the light jet mistag rate $f_l$
    and the $p_T$ resolution $\Delta p_T$, compared
    to the various scenarios that we discusse in this section.
\label{sec:variations}
  }
  \end{table}
%%%%%%%%%%%%%%%%%%%
