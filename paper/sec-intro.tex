\section{Introduction}

The measurement of double Higgs production will be one of the main
physics goals of the LHC program in its recently started high-energy
phase, as well as for its future high-luminosity phase, as
well as of the program of any future high-energy collider.
%
Double Higgs production is directly sensitive to the Higgs trilinear coupling, and thus 
provides information on the scalar potential responsible for electroweak symmetry breaking.
%
It is also sensitive to the underlying strength of the Higgs interactions
at high energies, and can test the composite nature of the 
Higgs boson~\cite{Giudice:2007fh,Contino:2010mh}.

In the Standard Model (SM), the dominant mechanism for the production of two Higgs bosons at the LHC is 
gluon fusion (see Ref.~\cite{baglio} and references therein), analogously to single Higgs production.
For a center-of-mass energy $\sqrt{s} = 14\,$TeV, the recently computed next-to-next to
leading order (NNLO)
total cross section is approximately $40\,$fb~\cite{deFlorian:2013jea}.
%
Feasibility studies in the case of a light Higgs boson have been performed for several different final states, including
$b\bar b\gamma\gamma$~\cite{Baur:2003gp,Barger:2013jfa},
$b\bar{b}\tau^+\tau^-$~\cite{Baur:2003gpa,Barr:2013tda,Dolan:2012rv,Dolan:2013rja},
$b\bar{b}W^+W^-$~\cite{Dolan:2012rv,Papaefstathiou:2012qe} and
$b\bar{b}b\bar{b}$~\cite{Baur:2003gpa,Dolan:2012rv,Gouzevitch:2013qca,Cooper:2013kia,Wardrope:2014kya,deLima:2014dta}.
%
While these various studies differ in their qualitative conclusions, it is clear
that the ultimate precision in the determination of the Higgs trilinear
will be provided by the combination of various final states.
%
Remarkably, searches for the production of Higgs pairs, whose rates
could be enhanced by BSM effects have already been performed
by ATLAS~\cite{Aad:2015uka} and CMS~\cite{Khachatryan:2015yea},
though the sensitivity of these searches is still far from
that required for a SM production mechanism.

The advantage of the $b\bar{b}b\bar{b}$ final state is the the signal yield
is enhanced by the large branching fraction of Higgs bosons into bottom-anti-bottom
pairs, ${\rm BR}\lp H\to b\bar{b}\rp\simeq 0.57$~\cite{Dittmaier:2012vm}.
%
However, this channel needs to deal with an overwhelming QCD multi-jet background,
which makes this measurement very challenging.
%
Previous studies of Higgs pair production in this
channel~\cite{Wardrope:2014kya,deLima:2014dta}, considering
only the $4b$ irreducible QCD background, estimate that with a luminosity of
3000 fb$^{-1}$ a signal significance of $S/\sqrt{B}=$AAA can be obtained.
%
However, as show in this work, neglecting other QCD backgrounds such as $bbjj$ and
$jjjj$, with some of these light jets mistagged as $b$-jets, cannot be neglected, making
this measurement even more difficult.

Despite these large backgrounds, by combining the contribution of different final state
topologies, resolved, boosted and intermediate, with a careful optimization
of the analysis strategy, we find that it is possible to obtain a reasonable
good signal discrimination.
%
To further enhance the significance of the measurements, signal and background events
are processed using a Multivariate Analysis (in particular a feed-forward neural
network), that further enhance $S\sqrt{B}$ up to a value of AAA.
%
Therefore, we are able to rescue the potential of this final state in a realistic
analysis setup.

One of the original motivations of this study was to provide guidance to the experiments
concerning which aspects of the measurement should be particularly optimized
to increase the signal significance in this channel.
%
In this respect, the study the dependence on our results on the $b$-tagging mistag
efficiency and the light and charm jet mistag rate, the jet energy and transverse
momentum resolution, or the Higgs invariant mass resolution.
%
In each of these cases, we determine the significance  $S/sqrt{B}$ of the
measurement in both conservative and optimistic assumptions of these
parameters.


The structure of this paper is the following.
%
In Sect.~\ref{mcgeneration} we discuss the generation of the signal
and background Monte Carlo samples.
%
Then in Sect.~\ref{sec:analysis}
we describe our analysis strategy, and in particular
the classification of each individual event into
different categories according to the event topology.
%
The results for our traditional cut-based analysis
are presented in Sect.~\ref{sec:results}, and in
Sect.~\ref{sec:mva} we show how we can enhance signal
significance by means of multivariate techniques.
%
Finally in Sect.~\ref{sec:conclusions} we conclude and outline
possible future developements for Higgs pair production
in the $b\bar{b}b\bar{b}$ final state.
