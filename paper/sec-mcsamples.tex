\section{Monte Carlo samples}
\label{mcgeneration}

First of all in this section we discuss the details of the
Monte Carlo generation of the signal and background
events used in this work.

\subsection{Signal}
%%%%%%%%%%%%%%%%%%%%%%%%%%

Higgs pair production is simulated at leading order using
{\tt MadGraph5\_aMC@NLO}~\cite{Alwall:2014hca}, with finite $m_t$ effects taken into
account.\footnote{We thank Eleni Vrodynou for assistance here.}
%
We use the NNPDF 3.0 $n_f = 4$ LO set~\cite{Ball:2014uwa} with
$\alpha_S(m_Z^2)=0.118$
as interfaced via {\tt LHAPDF6}~\cite{Buckley:2014ana}.
%
We use the default settings of the renormalization and factorization
scale of the model.
%
The Higgs boson parameters are also the same in the default model,
in particular we use $m_H=125$ GeV, consistent with the latest
measurements from ATLAS and CMS.

%
Parton level events are then showered with the {\tt Pythia8} Monte
Carlo~\cite{Sjostrand:2007gs,Sjostrand:2014zea}, in particular with {\tt v8.201}.
%
We use the default settings in the parton shower, in particular
we use the Monash 2013 tune~\cite{Skands:2014pea}, based on the NNPDF2.3LO PDF set~\cite{Ball:2012cx}.
%
We have switched off hadronisation, since this allows a more transparent modeling of
$b$-tagging.
%
A complete simulation of $b$-tagging at the hadron level is difficult to achieve without a full
detector simulator, which is beyond the scope of this paper.
%
For a similar reason, we don't consider underlying event, multiple parton interactions and
pile-up, which can be taken of using a variety of subtraction methods~\cite{Cacciari:2009dp}.


The signal samples are rescaled so that the integrated cross-section matches that of the
NNLO calculation~\cite{deFlorian:2013jea}: this corresponds to using a $K$-factor $\sigma_{NNLO}/sigma_{LO}=2.3$.
%
Let us note that resummed NNLO+NNLL calculations for Higgs pair production have also become available recently~\cite{deFlorian:2015moa},
leading to a moderate enhancement as compared to the fixed-order NNLO calculation.

\subsection{Background}

Now we turn to discuss the Monte Carlo generation of the relevant background processes.
%
All background samples are generated with the {\tt SHERPA} event generator~\cite{Gleisberg:2008ta}, v2.1.1.
%
As in the case of the signal generation, the NNPDF 3.0 $n_f = 4$ LO set with strong coupling $\alpha_S=0.118$ is used for all samples.
Factorisation and renormalisation scales are set as $\mu_F=\mu_R=H_T/2$.

Samples!!!!

At the generation level the following basic cuts are applied.
%
Each final state particle in the hard process must have $p_T \ge 20$ GeV, and be located within $| \eta | \le 3.0$. All final state particles must be separated by a minimum $\Delta R_{\mathrm{min}} =0.1$.
%
We have checked that the generator-level cuts are loose enough as compared to the actual
analysis cuts.
%
Total cross-sections and details of the samples generated are shown in Table~\ref{tab:samples}. 


%%%%%%%%%%%%%%%%%%%
\begin{table}[h]
\begin{center}
\begin{tabular}{|c|c|c|c|c|}
\hline
Process &  Generator & $N_{\mathrm{evt}}$ & $\sigma_{\mathrm{tot}}$ \\
\hline
\hline
$pp \to HH$ &  {\tt MadGraph5\_aMC@NLO} & 100K & $1.729\times10^{-2}$ pb \\
\hline
\hline
$pp \to b\bar{b}b\bar{b}$ &  {\tt SHERPA}v2.1.1 & 3M &$1.121 \times10^3$ pb \\
$pp \to b\bar{b}jj$ &  {\tt SHERPA}v2.1.1 & 3M & $2.659 \times 10^5$ pb \\
$pp \to jjjj$ &  {\tt SHERPA}v2.1.1 & 3M  & $9.709\times 10^6$ pb \\
$pp \to t\bar{t}$ &  {\tt SHERPA}v2.1.1 & 3M & $2.514\times 10^3$ pb \\
\hline
\end{tabular}
\caption{\small Summary of signal and background samples generated in this work,
together with the corresponding generator-level cross-section.} \label{tab:samples}
\end{center}
\end{table}%
%%%%%%%%%%%%%%%%%%%%%%%%%%%%%%%%%%%%%%%%%%%

As a cross-check of the background cross-sections reported in Table~\ref{tab:samples}, we have produced multi-jet samples
using the {\tt MadGraph5\_aMC@NLO} program, and compared with the results for the same processes reported in
Ref.~\cite{Alwall:2014hca}.
%
For comparison with the latter numbers, 
we require in all samples four anti-$k_T$ $R=0.5$ jets with $p_T \ge 80 $ GeV, and the leading jet must have $p_T \ge 100$ GeV.
%
All jets must be within an acceptance of $|\eta| \le 2.5 $.
%
In the case of the samples with $b$ quarks in the final state, these requirements are extended to the appropriate number of $b$-jets. For example, in the 2$b$2$j$ sample there must be at least two $b$-jets that pass the cuts outlined above.
%
We find agreement, within the scale uncertainties, between the {\tt MadGraph5\_aMC@NLO} and {\tt SHERPA} calculations of the multi-jet
backgrounds, so we can be confident that the set-up that we will use in this analysis is robust enough.

