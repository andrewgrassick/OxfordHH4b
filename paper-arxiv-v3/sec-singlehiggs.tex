
\appendix

\section{Single Higgs backgrounds}
\label{app:singlehiggs}

As discussed in Sect.~\ref{mcgeneration},
in our analysis we neglect single Higgs production processes,
since they are much smaller than both the signal and the main
QCD multijet backgrounds.
%
To explicitly demonstrate this, we have generated LO samples
using  {\tt MadGraph5\_aMC@NLO}
for the
following single-Higgs processes:
  \begin{enumerate}
  \item $Z(\to b\bar{b})h(\to b\bar{b})$ (electroweak)
  \item $t\bar{t}h(\to b\bar{b})$
    \item $b\bar{b}h(\to b\bar{b})$ (QCD)
  \end{enumerate}
  For each processes, we have  generated 1M events, and in
 Table~\ref{HK}  we list resulting the
  LO and NLO cross-sections at the generation level.
  %
  The subsequent decays and the
  corresponding branching fractions are not included in these cross-sections,
  since
  these are taken care by the {\tt Pythia8} parton shower.
  %
  The values of these branching fractions 
  are listed in Table~\ref{HBF}, corresponding
  to the  most recent averages from the PDG.
  %
  In the case of the $t\bar{t}h$ process, we
  consider only the fully hadronic decays
  of the top quark, since leptonic and semi-leptonic decays
  can be suppressed
  by means of a lepton veto.
  
  %%%%%%%%%%%%%%%%%%%%%%%%%%%%%%%%%%%%%%%%%%%%
  %%%%%%%%%%%%%%%%%%%%%%%%%%%%%%%%%%%%%%%%%%%%
  \begin{table}[h]
\begin{center}
\begin{tabular}{|c|c|c|c|}
\hline
Sample & LO & NLO & $K$-factor\\
\hline\hline
$Zh$ (13 TeV) & $6.5 \cdot 10^{-1}$ pb & $ 7.7 \cdot 10^{-1}$ pb & 1.19 \\
$t\bar{t}h$ (13 TeV) & $3.8 \cdot 10^{-1}$ pb & $4.6 \cdot 10^{-1}$ pb & 1.29 \\
$b\bar{b}h$ (13 TeV) &  $4.9 \cdot 10^{-1}$ pb & $6.1 \cdot 10^{-1}$ pb & 1.22 \\
\hline
\end{tabular}
\caption{\small LO and NLO cross-sections at the generation level for the single-Higgs background
  processes listed above, computed using {\tt MadGraph5\_aMC@NLO}.
  %
  The subsequent decays and the corresponding branching fractions are not included in these generation-level cross-sections. \label{HK}
}
\end{center}
  \end{table}%
  %%%%%%%%%%%%%%%%%%%%%%%%%%%%%%%%%%%%%%%%%%%%
  %%%%%%%%%%%%%%%%%%%%%%%%%%%%%%%%%%%%%%%%%%%%

 %%%%%%%%%%%%%%%%%%%%%%%%%%%%%%%%%%%%%%%%%%%%
  %%%%%%%%%%%%%%%%%%%%%%%%%%%%%%%%%%%%%%%%%%%%
  \begin{table}[h]
\begin{center}
\begin{tabular}{|c|c|c|c|}
\hline
Sample & Decay & Branching Fraction\\
\hline\hline
$Zh$ & ($Z\to b\bar{b}$)($h\to b\bar{b}$) & 0.086 \\
$t\bar{t}h$ & $(W\to q\bar{q})^2$($h\to b\bar{b}$) & 0.26 \\
$b\bar{h}h$ & $h\to b\bar{b}$ & 0.57 \\
\hline
\end{tabular}
\caption{\small The values of the branching fractions applied to the single-Higgs
  background processes from Table~\ref{HK}, corresponding to
  the most updated PDG values. \label{HBF}}
\end{center}
  \end{table}%
   %%%%%%%%%%%%%%%%%%%%%%%%%%%%%%%%%%%%%%%%%%%%
  %%%%%%%%%%%%%%%%%%%%%%%%%%%%%%%%%%%%%%%%%%%%

  In Table~\ref{HBxsec}
  we show the signal and background cross-sections at the end of the cut-based analysis, before the MVA is applied,
  in the case without PU.
  %
  We separate the results into the three exclusive categories used in our analysis.
  %
  From this comparison, we see that as expected, at the end of the cut-based analysis, the single-Higgs
  backgrounds are smaller than the QCD multijet background by several orders of magnitude.
  %
  In addition, we find that already at the end of the cut-based analysis the di-Higgs
  signal is also larger than all the single-Higgs backgrounds in all the selection categories.
  %
  Since this discrimination can only be improved by the MVA, we
  conclude that neglecting single-Higgs backgrounds is a reasonable
  approximation.
%
  From Table~\ref{HBxsec} we also observe that in the resolved
  and intermediate categories $Zh\to b\bar{b}b\bar{b}$ is
  the dominant single-Higgs background, while $t\bar{t}h(\to b\bar{b})$ is
  instead the most important one in the boosted category.
  
%%%%%%%%%%%%%%%%%%%%%%%%%%%%%%%
  \begin{table}[h]
\begin{center}
\begin{tabular}{|c|c|c|c|c|}
\hline
& Sample &  \multicolumn{3}{c|}{Pre-MVA cross-section (fb)}\\
 & &  Boosted  & Intermediate & Resolved \\[0.1cm]
\hline\hline
Signal & $hh\to b\bar{b}b\bar{b}$ & $3.5\cdot 10^{-1}$  & $2.2\cdot 10^{-1}$ &  $1.2\cdot 10^{0}$ \\[0.1cm]
\hline
\multirow{4}{*}{Backgrounds} & QCD multijet &  $2.5\cdot 10^{+2}$ & $1.8\cdot 10^{+2}$ & $4.9\cdot 10^{+3}$ \\
&$Z(\to b\bar{b})h(\to b\bar{b})$ & $2.0\cdot 10^{-2}$ & $1.2\cdot 10^{-1}$ & $7.5\cdot 10^{-1}$ \\
&$t\bar{t}h(\to b\bar{b})$ & $5.1\cdot 10^{-2}$ & $6.3\cdot 10^{-3}$ & $4.0\cdot 10^{-1}$ \\
&$b\bar{b}h(\to b\bar{b})$ & $2.3\cdot 10^{-3}$ & $5.5\cdot 10^{-3}$ & $2.6\cdot 10^{-1}$\\
\hline
\end{tabular}
\end{center}
\caption{\small \label{HBxsec} Signal and background cross-sections at the end of the cut-based analysis
  (before the MVA is applied), in the case without PU.
  %
  We separate the results into the three exclusive categories used in our analysis.
  %
}
  \end{table}%
  %%%%%%%%%%%%%%%%%%%%%%%%%%%%

